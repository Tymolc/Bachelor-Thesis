\section{Introduction}
\label{sec:intro}

\subsection{Motivation}
The trajectory around a track that allows a given vehicle to traverse the circuit in the minimum amount of time is called the ideal racing line. There are a lot of parameter that determine the minimum lap time. In general it is a trade-off between the length of the taken line and the speed carried around the track, which in most cases turns out to be the line with the least curvature.  
Needless to say, in a race this is the path every driver wants to take. In reality, however, this isn't an easy task. In order to achieve best times, a racing driver has to remember exactly which corner of the track comes next, how fast they can take this corner, see how the car is positioned on the track and process many more information.
To assist aspiring drivers feel like professionals, we implemented a racing driver analysis system, that is capable of recording, comparing and evaluating race related data. The system saves a complete history of many data the car produces, like speed, throttle position and safety relevant features, like engine temperature and tracks the car on the course using a combination of different methods.
This car tracking will be the main focus of this thesis. The comparison of two drivers requires very robust measurements, that return the same result for every given input.
The most common tracking method, GPS, relies on external sources, namely satellites that observe specific areas and communicate with the GPS device. The biggest problem with this traditional approach is, that GPS isn't fully available everywhere. Especially on remote areas, which is where most race tracks are, there tends to not be complete coverage. This would result in faulty results and incomparable racing lines.
In order to break away from these dependencies, we used a camera-based approach. This allows us to track the relative translation and rotation of the vehicle independent of its speed, location or surroundings.
Apart from the recording of these racing lines, this paper also discusses how to process the obtained data so that they provide a useful support for racing drivers.

\subsection{Project Scope}
During the one year long bachelor project "Feel the Car - Automatic Anomaly Detection for Unstructured Test Drive Data" at the Hasso-Plattner-Institute, Potsdam, I worked together with four fellow students and in cooperation with Mercedes-AMG on analyzing video and sensor data from car test drives.
Part of the project was to find a possibility to retrospectively evaluate a race for a given driver. This is especially useful for AMGs Driving Academy, which is a service that lets non-professional drivers drive on race tracks and become a racing driver. To achieve this goal, and other requirements, such as detecting potholes on the street or security relevant warning signs in races, we used a stereo camera and a mini-computer, that processed the incoming raw data.
Additionally, an On-Board-Diagnostic 2 (OBD-II) dongle was used to extract the sensor data that cars have already integrated.
The analyzed result was then displayed in our web application which also provided a driver overview so every driver had a detailed overview over their recorded laps.

\subsection{Problem Introduction/Description/Context}

\clearpage