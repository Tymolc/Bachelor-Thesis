\section{Implementation}
\label{sec:implementation}

Most of the image processing steps were implemented using OpenCV, an open-source C++ computer vision library. 

\subsection{Recording}
\subsubsection{Visual Odometry}
To extract the Visual Odometry data from the video stream, we used a library called libviso2, which was developed at the Karlsruhe Institute of Technology (KIT). The library needs a left and right hand, rectified image from a stereo camera. The ZED camera from Stereolabs we used already rectifies the images, so we only need to convert the received stereo image into two data buffers.
For that the left and right image is stored in each one OpenCV Mat, converted into grayscale and then transformed into a uchar buffer.
To receive the images from the camera, we created a CameraHelper class, which centrally grabs the frames and distributes them to the different modules, like the RacingLineDetector, FlagDetector, etc.
Now that the prerequisites are met, we can feed the images to our library.
After analyzing the input as described in section \ref{subsec:vo}, the output will be a 4x4-matrix, the so called pose.

\begin{table}[!ht]
 \begin{center}
  \begin{tabular}{c c c c}
   $R_{11}$ & $R_{21}$ & $R_{31}$ & $T_{x}$\\
   $R_{12}$ & $R_{22}$ & $R_{32}$ & $T_{y}$\\
   $R_{13}$ & $R_{23}$ & $R_{33}$ & $T_{z}$\\
   0 & 0 & 0 & 1
  \end{tabular}
 \end{center}
 \caption{Pose as returned by Visual Odometry algorithm}
\end{table}

The pose consists of the XYZ 3x3 rotation matrix as defined by all $R_{nm}$ and the XYZ translation vector $T_{x/y/z}$. 

\begin{lstlisting}
Mat pose; //position relative to last position
Mat motionMat; //position relative to starting point

while(!finished){
	//library processing
	pose = pose * motionMat;

	float diffX = (pose.at<float>(0, 3));
	float diffZ = (pose.at<float>(1, 3));
	Point2d position = Point2d(diffX, diffY);
	//JSON-object creation 
}
\end{lstlisting}

For our purposes the rotation matrix is not that important, because the racing line is only a series of dots. The rotation of the car doesn't influence the result, so all we need is to extract the translation vector, store it in a vector object and write it into a list. This object contains the summation of all translation vectors, making up the position relative to the starting point.
At this point it is simply a matter of connecting consecutive dots in the list and drawing the resulting lines. The result will be the recorded racing line.

\subsubsection{Sensordata}
The OBD-II dongle uses a bluetooth connection to communicate with our analysis software. Via this connection, which is created automatically at startup by the operating system of our processing unit, we can open a serial port to write and read data from.

Once the connection is established, the serial connection has to be prepared for the specific settings of the dongle.
A couple of control commands put it into the state that we can work with: 

\begin{lstlisting}
void SensorDataCollector::initialize(int fd) {
	waitForAnswer(fd, "ATZ");	//reset device
	waitForAnswer(fd, "ATL1");	//enable linefeed
	waitForAnswer(fd, "ATSP0");	//autodetect protocol
}
\end{lstlisting}

The function \textit{waitForAnswer} writes the request to the dongle and returns the answer to an optional buffer. For the initialization this buffer is irrelevant, but as soon as we want to receive data, it is necessary.

\begin{lstlisting}
void SensorDataCollector::
	waitForAnswer(int fd, char *buf, string request) {
	int nbuf = 0;
	int i = 0;
	const int TIMEOUT = 30;
	bool end_of_answer = false;

	request.append("\r"); //tells the dongle where the request ends
	long writtenBytes = write(fd, request.c_str(), strlen(request.c_str()));
	if (writtenBytes < 0) {
		//errorhandling
	}
	while (!end_of_answer && i < TIMEOUT) {
		long n = read(fd, buf + nbuf, sizeof buf); //read one char at a time
		nbuf += n;
		i++;
		end_of_answer = receivedAnswer(buf); //receivedAnswer checks if last character was a >
	}
}
\end{lstlisting}

Now, after determining which PIDs are supported by the car (using PID 0100), we constantly loop through all of them and calculate the corresponding value. Together with the description, this value is then sent to our server as a JSON object.

\subsection{Comparison}
\subsubsection{Google Maps}
In order to properly identify important parts of the track, we decided to connect our recorded racing line with Google Maps. As our recording approaches can't extract any geographic information, we need at least two GPS point as a reference. One GPS point isn't enough, because that wouldn't specify the direction the car traveled. As the points returned by VO are in meters, we can calculate the corresponding geo-coordinates in relation to that reference point. 

For that we need the starting latitude $\phi_1$, the starting longitude $\lambda_1$, the bearing $\theta$, which is the angle between two VO-points plus the displacement bearing set by the two reference GPS-coordinates, and the distance $\delta$ traveled, which is the distance between the two VO-points.

The resulting GPS-coordinates are calculated like this:

\begin{lstlisting}[language=JavaScript]
for (var k = 1; k < voData.length; k++) {
	calculatedGPS[k] = {};

	var d = Math.sqrt(Math.pow((voData[k].x - voData[0].x - xCorrection), 2) + Math.pow((voData[k].y - voData[0].y - yCorrection), 2)) * voScaleFactor / 1000;
	var stdVec = [0, 1];
	var pointVec = [voData[k].x - voData[0].x - xCorrection, voData[k].y - voData[0].y - yCorrection];

	var bearingOfVoPoint = ((stdVec[0]*pointVec[0])+(stdVec[1]*pointVec[1]))/(Math.sqrt(Math.pow(stdVec[0], 2)+Math.pow(stdVec[1], 2))*Math.sqrt(Math.pow(pointVec[0], 2)+Math.pow(pointVec[1], 2)));
	bearingOfVoPoint = Math.acos(bearingOfVoPoint);
	bearingOfVoPoint = (pointVec[0] >= 0) ? bearingOfVoPoint : (2*Math.PI)-bearingOfVoPoint;
	bearingOfVoPoint += rotation;

	var lat2 = Math.asin( Math.sin(lat1)*Math.cos(d/R) + Math.cos(lat1)*Math.sin(d/R)*Math.cos(bearingOfVoPoint) );
	var lng2 = lng1 + Math.atan2(Math.sin(bearingOfVoPoint)*Math.sin(d/R)*Math.cos(lat1), Math.cos(d/R)-Math.sin(lat1)*Math.sin(lat2));
	lat2 = deg(lat2);
	lng2 = deg(lng2);

	calculatedGMapsPoints.push(new google.maps.LatLng(lat2, lng2));
}
\end{lstlisting}

$\phi_2 = asin(sin \phi_1 * cos \delta + cos \phi_1 * sin \delta * cos \theta)$

$\lambda_2 = \lambda_1 + atan2(sin \theta * sin \delta * cos \phi_1, cos \delta - sin \phi_1 * sin \phi_2)$


\begin{itemize}
	\item Polyline Overlay
	\item hidden Polyline for calculations and hover function vs visible, colored Polyline	
\end{itemize}
\clearpage