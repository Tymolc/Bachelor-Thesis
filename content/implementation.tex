\section{Implementation}
\label{sec:implementation}

Most of the image processing steps were implemented using OpenCV, an open-source C++ computer vision library. 

\subsection{Recording}
\subsubsection{Visual Odometry}
To extract the Visual Odometry data from the video stream, we used a library called libviso2, which was developed at the Karlsruhe Insitute of Technology (KIT). The library needs a left and right hand, rectified image from a stereo camera. The ZED camera from Stereolabs we used already rectifies the images, so we only need to convert the received stereo image into two data buffers.
For that the left and right image is stored in each one OpenCV Mat, converted into grayscale and then transformed into a uchar buffer.
Now that the prerequisites are met, we can feed the images to our library.
After analyzing the input as described in section \ref{subsec:vo}, the output will be a 4x4-matrix, the so called pose.
\begin{figure}
\[
 \begin{matrix}
  R11 & R21 & R31 & Tx \\
  R12 & R22 & R32 & Ty \\
  R13 & R23 & R33 & Tz \\
  0 & 0 & 0 & 1
 \end{matrix}
\]
\caption{Pose as returned by visual odometry algorithm}
\end{figure}

\subsubsection{Sensordata}
The OBD-II dongle uses bluetooth connection to communicate with our analysis software. Via this connection we can open a serial port to write and read data from.
The ELM 327 standard determines the formatting of the request and of the return value.
The request consists of the required mode and a Parameter ID (PID), which stands for one specific data value. In mode 1 the dongle returns the current values, which is what we need. Other modes, for example, return the values since the last engine failure or information about the car.
Once the request is send to the dongle, a hexadecimal encoded value consisting of a specified number of Bytes will be returned. With a formula, which is also specified in the ELM 327 standard, this return sentence can be decoded to an integer number, which represents the real result.

\subsection{Comparison}
\subsubsection{Google Maps}
\begin{itemize}
	\item Polyline Overlay
	\item hidden Polyline for calculations and hover function vs visible, colored Polyline	
\end{itemize}
\clearpage