\documentclass[%
a4paper,
12pt,
2.5headlines, 
bigheadings, 
titlepage, 
openbib,
%draft
]{scrartcl}

%%% PACKAGES
\usepackage[ngerman, english]{babel}
%% FONTS


\usepackage[T1]{fontenc}
\usepackage{geometry}
\usepackage[utf8x]{inputenc}
\usepackage{mathpazo}
\usepackage{helvet}
\usepackage{courier}
\usepackage{eurosym}
\usepackage{amsmath}
\usepackage{courier}
\usepackage{scrpage2}
\usepackage{graphicx}
\usepackage{xcolor}
\usepackage{multirow}
\usepackage{varioref}
\usepackage{babelbib}
\usepackage{makeidx}
\usepackage{tabularx}
\usepackage{floatflt}
\usepackage[pdftex, colorlinks, linktocpage, linkcolor=black, citecolor=black, urlcolor=black]{hyperref}
\usepackage[linesnumbered]{algorithm2e}
\pagestyle{scrheadings}


\geometry{a4paper, top=55mm, left=40mm, right=35mm, bottom=40mm,
headsep=10mm, footskip=22mm}
\linespread {1.25}
%%% COMMANDS

	\newcommand{\theauthor}{Tim Oesterreich}
	\newcommand{\matrnr}{770496}
	\newcommand{\thetitle}{Racing Line Detection - Recording and Comparison of Racing Lines}
	\newcommand{\thesubtitle}{ Ideallinienerkennung - Aufnahme und Vergleich von Ideallinien}

%%% COLORS
\input{utils/hpicolors}

%%% OTHER INPUTS
\input{utils/commands}
\input{utils/environments}
\newcommand{\frontmatter}{\pagenumbering{roman}}
\newcommand{\mainmatter}{\pagenumbering{arabic}\setcounter{page}{1}}
%%% INCLUDE ONLY
\setlength{\parindent}{0cm}
\setlength{\parskip}{0.25cm}
%%% DOCUMENT
\begin{document}
	%%% HEADER AND FOOTTITLES
	%\selectlanguage{ngerman}
	\selectlanguage{english} % {ngerman}
	\automark{section}
	\ohead{\includegraphics[height=1.3cm,clip,viewport={0 60 250 180}]{utils/hpi_logo.pdf}}
	\chead{}
	\ihead{\headmark}
	\setheadsepline{1.0pt}[\color{hpigrey}]
	%%% TITLEPAGE
	\hypersetup{%
		pdftitle	= {\thetitle},
		pdfsubject	= {Bachelor Thesis},
		pdfauthor	= {\theauthor},
		pdfcreator	= {PDFLaTeX},
		pdfproducer	= {LaTeX with hyperref and thumbpdf}
			   }
	
		\newgeometry{margin=1in}
	\titlehead{
	%\parbox[b]{10cm}{\sffamily{\Large Hasso Plattner Institut}  \\Prof.~Dr.~Helmertstra�e~2-3 \\14482 Potsdam} 
	\centering
	\includegraphics[height=4cm]{utils/hpi_logo_text.pdf}
	
	}		\subject{{\LARGE Bachelor's Thesis}\\}
	\title{\thetitle}
	\subtitle{\thesubtitle}
	\author{{\small by}\\\textbf{\theauthor}}
	%\dedication{Widmung\\mit mehreren\\Zeilen.}
	\date{Potsdam, August 2016}
	\publishers{
		\textbf{Supervisor}\\
		\vskip1em
		Prof. Dr. Christoph Meinel,\\		
		Philipp Berger, Patrick Hennig\\
	
		\vskip2em
		\textbf{Internet-Technologies and Systems Group}
		}
	\frontmatter
	\maketitle
	\restoregeometry
	%\input{titlepage_german}
		
		
		
		
	\section*{Disclaimer}

I certify that the material contained in this dissertation is my own work and does not contain significant portions of unreferenced or unacknowledged material. I also warrant that the above statement applies to the implementation of the project and all associated documentation.\\\\
Hiermit versichere ich, dass diese Arbeit selbst\"{a}ndig verfasst wurde und dass keine anderen Quellen und Hilfsmittel als die angegebenen benutzt wurden. Diese Aussage trifft auch f\"{u}r alle Implementierungen und Dokumentationen im Rahmen dieses Projektes zu.

	\begin{flushleft}
	Potsdam, \today
	\end{flushleft}
	\begin{picture}(150,70)
		\put(0,15){\line(1,0){150}}
		\put(0,0){(\theauthor)}
	\end{picture}
	\clearpage
	
	%%% Abstract
	\myabstract
	{%
	% deutsche Zusammenfassung
	deutsche Zusammenfassung...
	}{%
	% englischer abstract
//TODO

	Motor racing is all about getting around a track as fast as possible. One of the most important and most driver-influenced steps in achieving best times is the racing line, especially hitting the ideal racing line through corners. Our race analysis system collects different car-based data to evaluate and compare the driving style of drivers which helps them to gain an advantage over the competitors.

	This paper especially focuses on the recording and comparison of racing lines, using different methods, like Visual Odometry and lane detection and use the extracted data to compare and analyze the driving behavior of different drivers.
	}
	
	%%% TOC
	\tableofcontents
	\clearpage
	%%% INCLUDES
	\mainmatter

	%%%%%%%
	%% Add content here !!! %%%
	\section{Introduction}
\label{sec:intro}

\subsection{Motivation}
The trajectory around a track that allows a given vehicle to traverse the circuit in the minimum amount of time is called the ideal racing line. Needless to say, in a race this is the path every driver wants to take. In reality, however, this isn't an easy task. In order to achieve best times, a racing driver has to remember exactly which corner of the track comes next, how fast they can take this corner, see how the car is positioned on the track and process many more information. 

\subsection{Project Scope}

\subsection{Problem Introduction/Description/Context}

\clearpage
	\section{Related Work}
\label{sec:related_work}

\subsection{Racing Line calculation}
Contrary to most racing sports, where a racing line is usually drawn by an expert, video games, especially from independent developers, tend to concentrate on calculating the racing line for the computer-controlled cars.

\subsubsection{Vamos Racing Simulator}
One example is the \textit{Vamos racing simulator}. It uses an iterative cur\-va\-ture-min\-i\-mi\-za\-tion technique, simulating spring-loaded hinges that are placed in the middle of the track. The lateral forces a car produces during cornering are used to simulate the opening or closing of these hinges, iteratively shaping the racing line. After a certain number of iterations the curve stabilizes. This happens when the force across all hinges and therefore the curvature of the racing line is close to minimal. After the calculation is done, the possible speed of the racing cars at every point on the track can be calculated. 

\subsubsection{RaceOptimal}
RaceOptimal is a website that offers calculated ideal racing lines for 4 different vehicles on a variety of tracks. The approach is, similarly to that of Vamos, iterative. However their focus lies on the physics of the cars. They use a Bézier-curve to approximate a smooth line across a circuit, based on predefined control points. Then the fastest possible speed for every point on the track is calculated, based on the curvature of the turn and the friction coefficient and mass of the car and limited by the cars top speed. After that the acceleration is adjusted to not exceed the power of the engine and capabilities of the tires. Lastly, the graph is adjusted to also include the capacity of the brakes and aerodynamic drag. 
This algorithm is used on a certain number of possible lines, the initial population. These solution then breed offspring, which are combinations of the initial parents, and are modified randomly, to get as many different racing lines as possible.
The best children are kept and breed again, bad solutions will be thrown away. This process is repeated until the result, being the lap time on the given track, stays consistent.

\clearpage
	\section{Algorithm/Concept}
\label{sec:algorithm}
As this Bachelor Thesis tackles two problems, the algorithms can be devided into two categories as well.
The first category consists of racing line recording concepts and the second one of ways to compare these recordings.

\subsection{Recording}
\subsubsection{Visual Odometry}
Visual Odometry (VO) is a concept that most camera-based robots use for navigation. It uses two consecutive camera frames to calculate the rotation and translation of the camera between them. The advantage over traditional tracking systems like GPS is, that it isn't dependent on any external sources, like satellites or radio towers, to determine the position of the object.
Also it is capable of much higher polling rates than most GPS receivers, which normally poll at 1 Hz, so they get 1 positional update per second. The potential speed of visual odometry is linked to the framerate of the camera. That way, a camera that records at 30 frames per second enables Visual Odometry to estimate a position 30 times a second. 
The crude algorithm behind VO determines and stores recognisable features in one frame, using a feature detection algorithm.
At first we need two consecutive, recitifed camera images. It is important that the images are rectified, because otherwise the edges of the image would be curved and produce wrong results.
Our soulution used the Harris corner detection method for determining features. Finding corners in an image is a very important prerequisite for the final position determination, because it allows us to calculate direction vector to another point. This other point being the same feature in the following camera frame, where the Harris detector is used, as well.
Having two sets of corners from either image we can track each feature from one frame to another. 
\begin{itemize}
    \item Lucas-Kanade?
    \item Jacobian?
\end{itemize}
Now that we have an end point to every (possible) start point, we can construct a set of directional vectors.
Removing outliers (i.e. vectors that are much longer or go in a vastly different direction than most vectors) an approximation translation and rotation the camera conducted during the two frames can be calculated.
Outlier removal is important, because otherwise the movement of other cars on the track would cause the algorithm to think the camera was moving in a different direction than it actually was.
If the taken path includes loops it is possible to optimize the result by using loop detection. If any features are redetected at a later point in time and the current position is close to the position the features were redetected, the path can be streched in a way that it connects to the earlier path.
This can reduce errors that accumulated due to small estimation errors in the previous steps.

%%%%%%%%%%%%%%%%%%%%%%%%%%%%%%%%%%%%%%%%%%%%%%%%%%%%%%%%%%%%%%%%%%%%%%%%%
%TODO:
%\begin{itemize}
%  \item track features from one image to another (Lucas-Kanade-Method)
%  \item approximate direction vector based on emerging vectors (RANSAC?)
%  \item optimize result (Loop-Detection, Algorithm)
%\end{itemize}

\subsubsection{Lane Detection}
Lane Detection uses markings on a road to determine its lateral boundaries in realtion to the recording camera. It is often used in autonomous driving and driving assistance systems, because the position of the car in between lane markings gives a lot of information about the direction the car is traveling. 
In our case, not only can we determine a traveling direction, but also the position on the lane.
By tracking the distance to the left and right lane, a deviation to the middle of the lane can be calculated. 
If the driven track is know or the camera is capable of determining a scale (e.g. by using a stereo camera), the deviation can even be expressed as a metric length, which is helpful for extracting additional information, like speed and acceleration, later on.

%TODO: Algorithms
\begin{itemize}
  \item possibility to improve recording results on known track
  \item track distance to left and right lane edge
  \item possibility to determine accurate horizontal position on track
\end{itemize}

\subsection{Comparing}
\subsubsection{Visual Comparison of Racing Line with help of sensor data}

\begin{itemize}
  \item overlay racing lines with different lap times
  \item highlight relevant locations, like corners, showing detailed information about throttle and brake behaviour and curvature through the corner
  \item fastest way through corner is tradeoff between distance travelled and curvature, with low curvature allowing the driver to carry more speed to exit of the corner
\end{itemize}

\subsubsection{Mathematical Calculation of ideal line given a certain track}
\begin{itemize}
  \item calculating smooth curves around an edged representation of the track using Bézier curves
  \item naively not an optimized track, but fair approximation
  \item simulating a chain of springloaded masses -> ideal line calculated by itterating multiple times and taking the line that uses the least force to keep masses in bounds of track (//this seems quite complex, not really sure how this works yet)
\end{itemize}

\subsubsection{Mathematical Comparison of two paths}
\begin{itemize}
  \item convert pair of consecutive points into list of vectors; each vector is distance between points and angle to x-axis
  \begin{itemize}
    \item double dx = endPoint.X - startPoint.X;
    \item double dy = endPoint.Y - startPoint.Y;
    \item double magnitude = Math.Sqrt((dx * dx) + (dy * dy));
    \item double direction = Math.Atan2(dy, dx) * (180 / Math.PI);
  \end{itemize}
  \item combine vectors that have the same direction, by generating a new vector with the direction of the old ones and the sum of their lenghts. This indicates a straight, which is not that interesting for comparison
  \item Now we can compare the angle for each section of a corner
\end{itemize}
\clearpage

	\section{Implementation}
\label{sec:implementation}

Most of the image processing steps were implemented using OpenCV, an open-source C++ computer vision library. 

\subsection{Recording}
\subsubsection{Visual Odometry}
To extract the Visual Odometry data from the video stream, we used a library called libviso2, which was developed at the Karlsruhe Institute of Technology (KIT). The library needs a left and right hand, rectified image from a stereo camera. The ZED camera from Stereolabs we used already rectifies the images, so we only need to convert the received stereo image into two data buffers.
For that the left and right image is stored in each one OpenCV Mat, converted into grayscale and then transformed into a uchar buffer.
Now that the prerequisites are met, we can feed the images to our library.
After analyzing the input as described in section \ref{subsec:vo}, the output will be a 4x4-matrix, the so called pose.

\begin{table}[!ht]
 \begin{center}
  \begin{tabular}{c c c c}
   $R_{11}$ & $R_{21}$ & $R_{31}$ & $T_{x}$\\
   $R_{12}$ & $R_{22}$ & $R_{32}$ & $T_{y}$\\
   $R_{13}$ & $R_{23}$ & $R_{33}$ & $T_{z}$\\
   0 & 0 & 0 & 1
  \end{tabular}
 \end{center}
 \caption{Pose as returned by Visual Odometry algorithm}
\end{table}

The pose consists of the XYZ 3x3 rotation matrix as defined by all $R_{nm}$ and the XYZ translation vector $T_{x/y/z}$. For our purposes the rotation matrix is not that important, because the racing line is only a series of dots. The rotation of the car doesn't influence the result, so all we need is to extract the translation vector, store it in a vector object and write it into a list. This object contains the summation of all translation vectors, making up the position relative to the starting point.
At this point it is simply a matter of connecting consecutive dots in the list and drawing the resulting lines. The result will be the recorded racing line.

\subsubsection{Sensordata}
The OBD-II dongle uses a bluetooth connection to communicate with our analysis software. Via this connection we can open a serial port to write and read data from.
The ELM 327 standard determines the formatting of the request and return value.

The request consists of the required mode and a Parameter ID (PID), which stands for one specific data value. In mode 1 the dongle returns the current values, which is what we need. Other modes, for example, return the values since the last engine failure or information about the car.

\begin{equation}
	\label{eq:obd_example}
		\underbrace{4}_{status code}
		\underbrace{1}_{mode}
		\underbrace{0D}_{PID (hex)}
		\underbrace{37}_{payload}
\end{equation}
\begin{center}
	OBDII example response, returning 55 km/h as the vehicle speed
\end{center}

Once the request is sent to the dongle, we read the returned hexadecimal encoded value into a buffer. The answer always ends with a >-sign, so we read one character at a time, until we find this symbol. To make sure that the request arrived correctly, the answer contains some extra data (\ref{eq:obd_example}). The first character is the status. Status 4 means the request was understood and a answer could be delivered. The second, third and fourth characters repeat the obtained request. After that, the payload containing the data is appended. With a formula, which is also specified in the ELM 327 standard, this return sentence can be decoded to an integer number, which represents the real result.

\begin{table}[!ht]
	\begin{center}
		\begin{tabularx}{\textwidth}{|c | X | c | c | c | c | c |}
			\hline
			PID & Size in Bytes & Description & Min value & Max value & Unit & Formula\\ \hline
			0C & 2 & Engine RPM & 0 & 16 383.75 & rpm & $\frac{256A + B}{4}$\\ \hline
			0D & 1 & Vehicle speed & 0 & 255 & km/h & $A$\\ \hline
		\end{tabularx}
	\end{center}
	\caption{Excerpt of the available OBD II PIDs, A stands for the first, B for the second Byte}
\end{table}

\subsection{Comparison}
\subsubsection{Google Maps}
In order to properly identify important parts of the track, we decided to connect our recorded racing line with Google Maps. As our recording approaches can't extract any geographic information, we need at least two GPS point as a reference. One GPS point isn't enough, because that wouldn't specify the direction the car traveled. As the points returned by VO are in meters, we can calculate the corresponding geo-coordinates in relation to that reference point. 

For that we need the starting latitude $\phi_1$, the starting longitude $\lambda_1$, the bearing $\theta$, which is the angle between two VO-points plus the displacement bearing set by the two reference GPS-coordinates, and the distance $\delta$ traveled, which is the distance between the two VO-points.

The resulting GPS-coordinates are calculated like this:

$\phi_2 = asin(sin \phi_1 * sin \delta + cos \phi_1 * sin \delta * cos \theta)$

$\lambda_2 = \lambda_1 + atan2(sin \theta * sin \delta * cos \phi_1, cos \delta - sin \phi_1 * sin \phi_2)$

\begin{itemize}
	\item Polyline Overlay
	\item hidden Polyline for calculations and hover function vs visible, colored Polyline	
\end{itemize}
\clearpage
	\section{Evaluation}
\label{sec:evaluation}

\subsection{Visual Odometry to record Racing Lines}
\begin{easylst}
Pro: 
&Many Datapoints (in theory up to Framerate positions per second)
&Capable of detecting fairly slight movements
&No need for external sensors (apart from camera which already exists)
&works in every environment (especially where there is no or poor GPS signal)

Con:
&uses fairly expensive operations
&&in praxis mobile-range computers will only reach up to 10 fps
&no constant speed (slows down in areas with many feature points (like forests))
&fairly error prone
&&rounding errors and inaccurate calculations add up and distort the racing line
\end{easylist}
\subsection{Usefulness of Racing Line Comparison}


\clearpage

	\section{Future Work}
\label{sec:future_work}

\subsection{Open Research Questions}
\subsubsection{Real Time Capabilities}
\subsubsection{Increasing Accuracy}

\subsection{Extensions}
\subsubsection{Gamification}
So far drivers already have a simple and accessible way of comparing each other, however our method only provides a direct relation between two drivers. Unless a driver looks at all other drivers racing lines, they can't certainly asses if they drove a particularly fast lap or if the opponent was just slow in general.\\
Thats why a form of gamification could be implemented. Based on different factors, like cornering, control and boldness, a score could be determined that assigns a level to a driver. That way drivers could compare, even if they didn't drive on the same track and races could include drivers of similar skill levels to make them as intense as possible.

\subsubsection{Real-Time Comparison}
Currently, the recording of the racing line can not be achieved completely in real-time with hardware that is power-conscious enough to run with the power provided by a cars battery, as the most powerful hardware that can be considered to fulfill this requirement is the NVidia Jetson TX1, which we used. However, current implementations of Visual Odometry tend to not use the full capabilities of the device, namely they only run on the CPU, while the GPU is running idle. If, either using more powerful hardware or very specifically optimized algorithms, VO could be implemented to run in real-time, i.e. at 30-60 frames per second, this would make it possible to determine an exact location of the car at any point of the race, without the delay of sending the video data to a central, powerful processing unit or the inaccuracy of having to skip frames.\\
The position could then be transmitted directly to other cars and their drivers to inform them about how they compete in relation to all other drivers. Such an overview could even be shown in a heads-up display directly on the windscreen or, in order to not disturb the driver too much, by informing the team manager of the driver, who can then tell the driver.\\
The advantage of this process is, that no expensive measurement equipment and, especially, no external dependencies (apart from the communication between the cars) are needed. So it could be used anywhere.

\clearpage
	\section{Conclusion}
\label{sec:conclusion}
In order to develop an accessible racing driver analysis we explored a variety of methods to record and assess the driving behavior of racers. As our system should have as little external dependencies as possible, we looked for methods to do all of the analysis inside the car, which is why camera-based methods were used to record the racing line and race track. 

Additionally, we used the cars on-board sensor data to obtain further information about the driving and consequently enable us to answer complex questions, like the ideal braking and acceleration points on the track.

This made it possible to compare two drivers and discover their strengths and weaknesses and how to tackle them to become a better racing driver. Alternatively, an ideal racing line could be calculated, even without prior knowledge of the track with the help of lane detection. This could help even professional drivers improve and achieve best times.

Further improvements can be made through the addition of a gamification system, that gives a level to a driver based on their skill. Also with the help of a more optimized implementation a real-time comparison could be achieved, that could tell a driver at any time how they compare to another driver, e.g. displaying it at a heads-up display directly on the screen.
\clearpage

	%%%%%%%%
	
	%%% BIBLIOGRAPHY
	%\bibliographystyle{babunsrt3-fl}
	\addcontentsline{toc}{section}{Bibliography}
	\bibliographystyle{babunsrt-fl}
	\bibliography{projektbib}
	\clearpage	
	
	\input{content/appendix.tex}
	
	
\end{document}
